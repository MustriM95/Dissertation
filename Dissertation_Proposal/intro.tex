\chapter{Introduction and Background}

\section{Context and Justification}
Ecosystems are home to an enormous array of complex interactions, many of which are poorly understood, yet have important implications for how organisms grow and the functions they perform in the environment. One of the main challenges hindering our ability to describe and study ecological interactions stems from the heterarchical nature of the underlying mechanisms, which range across vast spatial and temporal scales. For instance, much in the same way that abiotic factors drive changes in the composition and functioning of organisms, organisms can modify their environment as a byproduct of their activities, creating non-linear feedback loops that span multiple levels of organization \cite{scheffer2001a, meacock2023a, philippot2023a}. Hence, it has become increasingly important to develop frameworks that reconcile how fundamental biological constraints and emergent macroscopic teleology shape ecological interactions. 

Traditional approaches to parsing the strength and direction of ecological interactions have relied heavily on correlational methods such as species co-occurrence networks and climate-based distribution models. Though these have been unquestionably useful, there is reason to doubt their applicability beyond the narrow range of conditions within which the relevant data were obtained \cite{barnes2022a, goberna2022a, pinto2022a}. An obvious implication for this uncertainty is the inadequacy of (mechanism-agnostic) statistical models for predicting potential climate change impacts. A more fundamental limitation is that, while excellent at recognizing patterns, these models provide minimal information about the universal mechanisms shaping ecological organization \cite{machado_modeling_2011, van_den_berg_ecological_2022}. 

An alternative approach lies in bridging the relationship between small-scale processes (e.g. genes, cells and organisms) and the macroscopic properties of populations and communities (e.g. energy processing and carrying capacity). Such process-based models have proliferated considerably \cite{pilowsky_process-explicit_2022}, however, the focus has been predominantly on making quantitative predictions, as opposed to illuminating fundamental principles \cite{marquet2014a}. While more reliable predictions are undeniably desirable, they typically come at the price of increasing and often intractable complexity, motivating the need for simple insights to constrain degrees of freedom. Of course, a complete understanding of ecological organization requires both approaches; detailed, process-explicit models allow us to forecast an ecosystem's likely trajectory, while first principles facilitate a more general synthesis of the driving forces. 

The following proposal explores the general question of how trait distributions in a given ecosystem arise as a function of environmental drivers and biophysical constraints. Furthermore, given these relationships, how can we use traits to infer an ecosystem's current and future condition?  This line of questioning dates at least as far back as Hutchinson's Homage to Santa Rosalía \cite{hutchinson1959a}. At that time focus centered on explaining species diversity along with its impact on stability and resilience \cite{loreau2010a}. In recent years there has been a general acknowledgment that any comprehensive understanding of biodiversity must consider not only the identity of species, but more importantly their traits \cite{enquist2015a, lavorel_predicting_2002, petchey_functional_2006}. Trait-based approaches have since flourished, offering a more detailed understanding of ecological niches, and engendering frameworks which frame old theory in observable terms \cite{chac2023a}. In my dissertation, I present four projects that aim to integrate plant and microbial functional traits into a wider theory of species interactions and ecosystem functioning. 

\section{Key Concepts}

\subsection{Functional Traits}

Trait based ecology (or TBE) seeks to understand how phenotypic diversity coordinates with the environment to shape population dynamics, community assembly and ecosystem functioning \cite{chac2023a}. The backbone of TBE is the assumption that the functions organisms carry out in a particular environment stem from observable traits. In other words functional traits directly map on to an organism's ability to grow, survive, and reproduce \cite{grime1988a}. Thus, the distribution of these functional traits is, in principle, reflective of favorable growth strategies and ongoing ecological processes \cite{norberg_phenotypic_2001}. While not formulating hypotheses in and of itself, TBE provides a framework to directly translate theory into quantitative predictions based on the shape and spatio-temporal evolution of functional trait distributions \cite{enquist2015a}.

Despite the apparent simplicity of TBE, its application requires careful consideration of which traits to focus on. While the practical details of trait selection have been discussed elsewhere \cite{chac2023a}, it is necessary to point out that, in most cases, the mechanistic link between trait values and performance is unknown. Additionally, it is relatively common to find that a given function is the result of multiple interacting traits, which need not interact linearly. 

\subsection{Metabolic Scaling}

Body size has long been recognized as a key organismal characteristic, having clear relationships with individual metabolic rates, population size, and appearing in many central macroecological patterns \cite{brown2004a, allen2005a, hatton2019a}. These relationships are attributed to the power law scaling of metabolic rate, which arises from biophysical constraints placed on the structure of distribution networks at the organismal level \cite{west_general_1997, brown2004a, savage_sizing_2008}. More importantly, these constraints are so tightly conserved throughout large portions of the tree of life that they are apparent across several orders of magnitude and shape processes on multiple levels of organization \cite{hatton2019a}. This generality makes MST exceptionally useful when attempting to relate the small scale properties of ecosystems to its extensive features because it effectively collapses two important axes of variation: body size and energy usage.

Though MST addresses how metabolic rates scale with body size, it did not, at its conception, successfully explain observed differences in the baseline rates (or normalization constants). Indeed, one of the major challenges faced by MST was the 20-fold variation in normalization constants across taxonomic groups \cite{brown2004a}. Much of this residual variation has been explained through the temperature dependence of metabolism, which accounts for the discrepancies between core phylogenetic clades (e.g. between ectotherms and endotherms) \cite{savage_sizing_2008, smith_systematic_2021, cook_id_thermodynamic_2021}. Additionally, ontogenetic growth and resource limitation appear to have important implications for much of the remaining variation within groups, though the exact relationship is not as straightforward as with temperature \cite{huang_water_2020}. It is quite clear, however, that baseline metabolic rates are closely coupled to specific aspects of life history strategies, namely, those that are independent of body size and temperature.

Finally, many of the assumptions held by MST do not necessarily hold at the smallest scales, as is evident from the inconsistent scaling of prokaryotes and unicellular eukaryotes \cite{delong_shifts_2010}. This issue has become the focus of recent attention, and current research points towards transitions in the constraints dominating growth as organisms become smaller. Notably, microbial metabolism seems to be disproportionately sensitive to thermal variation and matrix pH. Moreover, how microbes respond to temperature and pH varies tremendously across taxa \cite{jin2018a, smith_systematic_2021, garc2023a}.

\subsection{Plant life history strategies and the fast-slow spectrum}

The search for mechanisms that facilitate the enormous diversity of terrestrial plants has rendered a rich literature on the importance of life-history strategies. Life-history strategies have often been posed in terms of specialized phenotypes suited to varying levels of resource availability, disturbance, and stress. Each phenotype is typified by a correlated set of physiological and morphological traits that are, in turn, linked to specific environmental conditions \cite{grime1988a}. Furthermore, the trait correlations underpinning life-history strategies arise from differential allocation into basic functions such as: growth, reproduction, efficient assimilation of nutrients, competition, and stress tolerance \cite{tilman_constraints_1990, mooney_carbon_1972}. 

Global studies of plant traits have revealed striking consistency in patterns of variation, suggesting a set of physiological trade-offs that explain the linkage between phenotypes and life history strategies. Much work has since been done to reformulate life history theory in terms of axes of functional trait variation, offering new insights into the underlying constraints \cite{wright2004a, reich2014a, diaz_global_2016}.

One of the best studied aspects of the phenotype-strategy relationship is known as the Leaf Economics Spectrum (LES), which quantifies the rate of return on investment into photosynthetic tissues, going from fast to slow \cite{wright2004a}. The LES consists of six key traits: leaf mass per area, photosynthetic capacity, leaf nitrogen and phosphorus, dark respiration rate, and leaf lifespan. The relationship between these traits is largely attributed to optimization of carbon assimilation at the individual leaf level. For example, fast traits are characterized by low structural costs, short life spans, and a relatively large allocation to photosynthetic capacity \cite{kikuzawa_cost-benefit_1991, wang2023a}. Although the LES explains a large proportion of the variation in leaf traits, it does not account for relationships with non-photosynthetic tissues, nor does it explain observed correlations between leaf traits and climate \cite{jensen_physical_2013, bin_leaf_2022, Wieczynski2019}. 

The fact that LES traits co-vary with body size and climate points to a more comprehensive set of trade-offs which emerge as individual leaves are scaled to the organismal level. The precise origins of these trade-offs are still poorly understood, but evidence points to the importance of whole plant hydraulics (root and stem traits), dispersal, and nutrient limitation \cite{price_flow_2022, baraloto_decoupled_2010}.

\subsection{Microbial metabolism and life history strategies}

%A paragraph on microbial metabolism
Our understanding of microbial populations largely stems from the study of a small number of taxa that are amenable to lab culturing \cite{rinke_insights_2013}. Additionally, our limited ability to directly observe microbial life has forced us towards macroscopic descriptions of their activities \cite{prosser2007a}. Conversely, the relative simplicity of bacterial and archaeal physiology is well suited to quantitative descriptions of the underlying biochemistry \cite{manzoni2012a}. This combination of circumstances has led to a fruitful literature on microbial metabolism and thermodynamics and a proportionally poor grasp on the mechanisms controlling microbial communitiy assembly \cite{van_den_berg_ecological_2022}. 

Microbial life history theory is a rapidly emerging field, borrowing much of the general framework developed in the study of plants \cite{malik2020a}. Unlike plants, the key traits and strategies that govern microbial systems are poorly described. Existing theory has leveraged trade-offs between growth rate, bioenergetic yield and, to a smaller extent, stress tolerance \cite{contois1959a, manhart_growth_2018}. However, a detailed account of how these tradeoffs emerge is missing due to our limited ability to measure at the relevant spatio-temporal scales. Furthermore, this literature largely ignores how traits correlate with microbial metabolic pathways which are a crucial confound due to their ample diversity \cite{manzoni2012a, thompson2017a}.

Despite these challenges, the rapid development of various molecular and single cell imaging technologies have made it feasible to observe and characterize microbial phenotypes at subcellular resolutions \cite{abbasian2018a, jafarpour2018a}. Likewise, large databases of bacterial and archaeal genomes continue to spur the proliferation of gene-informed microbial models. Such models translate the presence of enzyme-encoding genes to reconstruct the potential metabolic pathways of microbial taxa \cite{mccarty2007a, karaoz2022a}.

\section{Study system}

My dissertation will not focus on a specific study system, but will leverage existing plant and microbial trait data along with climate models and remote sensing. For example, chapters 2 and 4 are largely centered on Stordalen Mire and the corresponding work carried out by the EMERGE Biological Integration Institute, a long term research project in northern Sweden \cite{hough2020a} (Hough et al., 2020, 2022; Freire-Zapata et al., 2024). Chapters 1 and 2 will rely on global trait data (GloP, KBase, BIEN, etc.) remote sensing (GEDI, SRTM), and climate models (CHELSEA).

\section{Approach}

The approach throughout all of my chapters is the synthesis and further development of existing theory. While the details vary considerably between chapters, the central thesis is that biological systems at all scales effectively optimize energy and information processing in some form. Additionally, given a set of biophysical constraints, it is possible to relate trait (phenotype) distributions to the proposed optimization target (for examples see \cite{west1999a,krakauer2020a,harte2022a}).

Broadly speaking, constraint based modeling assumes that a portion of the underlying parameters are inherently interdependent. By incorporating the corresponding symmetries (or constraints) into the model, it becomes possible to reduce parametric complexity and, in some cases, make straightforward deductions from the resulting equations. In some scenarios constraints do not provide sufficient information and further steps are required in order to make generalizations. This becomes especially relevant when dealing with many variable systems, such as diverse microbial communities and spatially structured environments. To address this issue I will be adopting various methods to approximate whole system dynamics from statistical moments.

In summary, my dissertation aims to combine microbial and plant life history theory with metabolic scaling to build models that link trait distributions to ecosystem functioning. In my first chapter I will extend leaf economics, incorporating whole plant hydraulics through allometries. My second chapter explores microbial communities from a thermodynamic perspective, leveraging our knowledge of microbial biochemistry to link the activities of key functional groups to the flow of energy and nutrients in a permafrost system. The third chapter will use size structured models to relate ecosystem size spectra to resilience and robustness, given different disturbance scenarios. Finally, chapter four investigates the enrgetic links between plant life history and the energetic favorability of plant derived soil inputs. In concert, this work will provide a theoretical basis for bridging patterns of microbial and plant functional trait variation in a given ecosystem, to its current and future states.


