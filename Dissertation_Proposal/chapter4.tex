\chapter{Metabolomic imprint of plant life history strategies}


Idea: Many plant functional traits are tightly correlated with specific life history strategies which impact how plants grow and reproduce. Some of these traits concern the stoichiometry and composition of different plant tissues (e.g. carbon:nitrogen and carbon:phosphorus ratios), showing some promising trends. In this chapter I am proposing to look a little bit closer at the chemistry of plant tissues, try and parse out what they are composed of, how many carbons, hydrogens, nitrogens, aromatic rings, etc. Ultimately, I want to see if there is a relationship between the lability/recalcitrance of plant tissues/exudates and their particular life history strategies.
Question: Do plant life history strategies dictate the free energy availability of plant litter and root exudates?
Why is it interesting? I think it is interesting because, though the relationship between life history and the stoichiometry of plant tissues is more or less well characterized, we don’t know the specific compounds that are driving the large scale trends. By looking at the structure of plant tissues and metabolites we could likely find trends at a finer scale, such as the size of molecules, their aromaticity, and the nominal oxidation state of carbon.

\section{Background}

%First paragraph introducing the problem: how can we tie above ground productivity to below ground carbon dynamics? How do plant communities impact microbial life in their immediate suroundings, and vice versa? Hint at plant functional traits and the leaf economics spectrum as a potential theoretical link. 

Carbon and nutrient cycling in terrestrial ecosystems is predominantly driven by intricate relationships between above-ground primary producers and soil microorganisms. These relationships, though strongly mediated by soil properties and local abiotic factors, airse through a series of biochemical transformations ranging from photosynthesis in plants to numerous metabolic reactions carried out by microbes. If we consider that plants and their corresponding soil inputs are the predominant source of organic carbon in terrestrial ecosystems, it follows that the range of fesible metabolic activities is generally constrained by the quantity and quality of plant derived organic matter. Furthermore, given the systematic covariation between environmental conditions, plant functional types, and plant litter composition \textcolor{red}{citations}; we should observe consistent relationships linking above ground plant communities to below ground microbial decomposition. Although there is evidence to support the existence of these relationships \textcolor{red}{citations}, our understanding remains largely descriptive and lacks a consistent framework grounded in plant energetics and biochemistry. Here I explore a new method of framing the biochemical ramifications of plant functional traits and life history theory with the aim of refining our predictions for soil carbon and nutrient cycling. 

% Second paragraph going into detail on the stoichiometric implications of the LES and life history theory, name a few of the main hypoetheses in this sphere, cite relevant papers.

As mentioned in prior chapters, plant life history strategies have important implications for numerous features of plant form and function, not least of which is the stoichiometry of various plant tissues \cite{reich2014a,wright2004a}. Hence, the chemical composition of plant derived organic matter can be tied to the preponderance of plant functional types via economic and energetic principles. For example, the LES predicts that faster growing species will have a higher mass-normalized leaf content of Nitrogen and Phosphorous relative to slower growing species, mainly due to the high N and P concentrations of metabolically active proteins \cite{diaz_global_2016}.Meanwhile the foliar ratio of N:P should show a negative correlation with maximal relative growth rate, which is attributed to the elevated demand for protein synthesis and the corresponding enrichment in ribosomal RNA, ribosomal DNA, and the length increase of intergenic spacer regions \cite{gusewell_n_2004, elser_biological_2000}. Despite the consistency between observed trends in plant stoichiometry and the cellular mechanisms proposed to explain them , the precise drivers remain difficult to identify given the large degree of inter- and intraspecific variation, as well as confounding effects arising from ecological interactions (e.g. herbivory and soil nutrient content).

% Third paragraph intoduces the how the LES and life history theory could have deeper ramifications regarding soil microbial activity. Again, introduce some of the well known ideas and cite important papers.

Now, turning to the impact of plant chemical composition on downstream microbial activity, it is quite clear that elemental ratios reflect the preponderance of key limiting processes, such as the link between C:N ratios and decomposability \cite{bakker_leaf_2011}. For example, Manzoni et al. found that low carbon:nutriet ratios (C:N and C:P) effectively reduce the carbon use efficiency of decomposers \cite{manzoni_stoichiometric_2010}. However, nutrient contents -- in terms of coarse grained elemental ratios -- do not always accurately predict the subsequent carbon dynamics due to potential constraints affecting bioavailability, such as the environmental concentration of terminal electron acceptors (TEAs). In scenarios where substrate decomposition is strongly modulated by TEA availability (e.g. wetlands) it is more infromative to consider measures of thermodynamic favorability as these directly relate to the energetic yield of substrates upon oxidation \cite{kleerebezem2010a,hough2020a}. In light of this, we might wonder: how do plant functional types relate to the thermodynamic favorability of the corresponding litter?

% Final paragraph of intro, give a rough idea of how the relationship between plant litter stoichiometry and functional traits can be expanded upon to infer further details on the decomposability of above ground carbon inputs. Elaborate on how metabolomic metrics such as NOSC can be used to improve our predictions of the linkage between plant and soil communities.

This final chapter explores how plant form and function, where functional traits fall on the spectrum of life history strategies, impacts the thermodynamic favorability of plant derived carbon inputs as inferred from the corresponding metabolomic signature. Initially, this will imply extending the theory of leaf economics to include predictions for how key metabolite concentrations (namely, structural polymers and primary metabolites) relate to the different axes of the LES. Successively, I aim to use predicted metabolite balances to justify a theoretical link between life history strategies and the thermodynamic favorability of photosynthetic tissues. 

\section{Approach and Methods}

% First paragraph is a summary of the overall approach. Firs elaborate on the extent of the theory, how will you modify current theory to include energetic measures such as NOSC. Next, mention that you will use metabolomics and EMERGE data to test these ideas. Summarize, in a succinct sentence, state what this work will accomplish.

% subsection: Elaborate theory of PFT and litter bioavailability    

% subsection: Propose method of testing predicitons with EMERGE (or other) datasets
